\documentclass[reprint,amsmath,amssymb,aps]{revtex4-2}
\usepackage{graphicx}
\usepackage{hyperref}

\begin{document}

\title{Complejidad, Retrocausalidad y Gravedad: Una Vía hacia la Unificación Física}
\author{Tu Nombre}
\affiliation{Institución o Departamento}
\date{\today}

\begin{abstract}
Se introduce un marco conceptual donde la complejidad y la retrocausalidad median la transición entre descripciones cuánticas y clásicas. La medición se modela como una proyección clásica inducida por luz con propagación causal y retrocausal simultánea, haciendo del colapso un efecto ilusorio. Se propone que los campos extensos portan anticomplejidad y que las trayectorias cuánticas múltiples son compatibilidades geométricas retrocausales en un espacio complejo acoplado. La cuenta dimensional conduce naturalmente a 11 dimensiones, proporcionando una conexión falsable con la Teoría de Cuerdas.
\end{abstract}

\maketitle

\section{Introducción}
Las tensiones entre relatividad general, mecánica cuántica y cosmología persisten en aspectos conceptuales clave: la dualidad onda-partícula, la expansión aparente del universo y la falta de falsabilidad directa de la Teoría de Cuerdas. Este trabajo propone un marco en el que complejidad y retrocausalidad articulan estas tensiones sin invalidar resultados empíricos establecidos.

\section{Magnitudes y espacios}
\subsection{Densidad de complejidad}
\begin{equation}
\rho_c(\mathbf{x},t) = \frac{N_{\text{estructuras heterogéneas}}(\mathbf{x},t)}{V_{\text{local}}}
\end{equation}
Se asume un espacio complejo $\mathcal{M}_C$ acoplado al espacio-tiempo $\mathcal{M}_{1,3}$, con métrica efectiva $g_C$ cuyo curvatura depende de $\rho_c$.

\subsection{Anticomplejidad de campos}
Los campos cuánticos extensos, al vehicular efectos retrocausales, portan anticomplejidad $\bar{\rho}_c$, complementaria de $\rho_c$. La simetría global materia–antimateria se reinterpreta como balance complejidad–anticomplejidad \cite{Susskind2018, Maldacena1998}.

\section{Luz, dualidad temporal y medición}
\subsection{Propagación bidireccional de la luz}
\begin{equation}
K_{\text{eff}} = \alpha(\rho_c) K_{\text{ret}} + \beta(\rho_c) K_{\text{adv}}
\end{equation}
con coeficientes dependientes de la densidad de complejidad del entorno.

\subsection{Sistema generador de representaciones clásicas}
\begin{equation}
\Pi_{\mathcal{O}}: \mathcal{H} \times \mathcal{M}_C \to \Omega_{\text{cl}}
\end{equation}
proyecta estados cuánticos $\psi$ en resultados clásicos $o \in \Omega_{\text{cl}}$, dependiendo del campo reactivo $\mathcal{R}_\mathcal{O}$.

\subsection{Colapso ilusorio}
El colapso es un efecto ilusorio; la conexión causal y retrocausal mediada por la luz selecciona trayectorias compatibles y proyecta $\psi$ en $o$ mediante $\Pi_{\mathcal{O}}$ \cite{Cramer1986, Aharonov1964}.

\section{Onda retrocausal en la cuántica}
\begin{equation}
\mathcal{A}(\psi_f, \psi_i|\rho_c) = \int_{\Gamma} \exp\left\{\frac{i}{\hbar}\left[S[\gamma] + \Phi(\gamma;\rho_c,g_C)\right]\right\} \mathcal{D}\gamma
\end{equation}
con $\Phi$ capturando el acoplamiento retrocausal. La contribución $\Phi$ desplaza fases relativas entre trayectorias sin introducir ruido estocástico.

\section{Multitrayectoria determinista}
\begin{equation}
\Gamma_{\text{comp}}(\psi_i;\rho_c, \mathcal{O}) = \{\gamma \in \Gamma : \delta (S+\Phi)=0 \text{ y }\gamma \text{ es estacionaria en } g_C \}
\end{equation}
La medida de compatibilidad:
\begin{equation}
\mathbb{P}(o|\psi_i,\mathcal{O}) \propto \mu(\Gamma_{\text{comp}}^{(o)})
\end{equation}

\section{Dimensiones y conexión con Teoría de Cuerdas}
Dos conteos equivalentes hacia 11 dimensiones:  
(i) $3$ masivas + $3$ complejas + $3$ cuánticas + $1$ tiempo + $1$ vacío,  
(ii) $3$ masivas + $3$ complejas + $3$ cuánticas + $2$ temporales (causalidad y retrocausalidad) \cite{Greene2000, Penrose2004}.

\section{Apéndice especulativo: Gravedad dual y cosmología}
La gravedad exhibe dualidad:  
- Modo partícula (causal): acumulación local de masa.  
- Modo onda (retrocausal): expansión global del espacio-tiempo \cite{Carroll2010}.  

\section{Discusión y falsabilidad}
Predicciones: interferometría sensible a $\rho_c$, medición adaptativa y correlaciones retrocausales débiles en señales de fondo. El marco asigna significado físico a 11 dimensiones y propone vías de falsación.

\bibliographystyle{apsrev4-2}
\bibliography{referencias}

\end{document}
